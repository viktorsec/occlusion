\chapter*{Úvod}
\addcontentsline{toc}{chapter}{Úvod}
Rozšírená realita, teda počítačom obohatený pohľad na skutočný svet nachádza čoraz väčšie uplatnenie v zábave, medicíne, armáde, reklame a mnohých ďalších oblastiach ľudskej činnosti, najmä preto, že sa rozširujú hardvérové možnosti. To, na čo bolo kedysi treba drahé laboratórne vybavenie, ako výkonné počítače, profesionálne kamery a senzory, dnes dokáže takmer každý moderný mobilný telefón s kamerou. Rozšírená realita otvára nové možnosti interakcie medzi virtuálnym a fyzickým svetom, čo môže byť v budúcnosti veľmi zaujímavé.

O tejto téme je počuť čoraz viac a stojí za to sa ňou zaoberať, nakoľko jej aplikácie môžu byť veľmi užitočné, ako ukážeme neskôr. V práci rozoberáme jednotlivé možnosti aplikácii tejto technológie aj s konkrétnymi príkladmi.

V jadre práce pojednávame o niektorých metódach registrácie obrazu, používaných na vytvorenie rozšírenej reality a popisujeme krok za krokom ako sme vytvorili demo aplikáciu rozšírenú o oklúziu s reálnymi objektmi. Popisujeme načítanie modelov, registráciu scény, niekoľko spôsobov získania modelu oklúdera, kalibráciu kamery a samotné vykresľovanie oklúzie. Na konci kapitoly prezentujeme výsledky demo aplikácie.

Práca je ukončená krátkym pojednaním o perspektíve budúceho využitia tejto technológie a o tom, čo bude potrebné pre to, aby sa používanie rozšírenej reality
rozšírilo.